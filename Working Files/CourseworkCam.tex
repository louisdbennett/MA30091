% Options for packages loaded elsewhere
\PassOptionsToPackage{unicode}{hyperref}
\PassOptionsToPackage{hyphens}{url}
%
\documentclass[
  11pt,
]{article}
\usepackage{amsmath,amssymb}
\usepackage{lmodern}
\usepackage{iftex}
\ifPDFTeX
  \usepackage[T1]{fontenc}
  \usepackage[utf8]{inputenc}
  \usepackage{textcomp} % provide euro and other symbols
\else % if luatex or xetex
  \usepackage{unicode-math}
  \defaultfontfeatures{Scale=MatchLowercase}
  \defaultfontfeatures[\rmfamily]{Ligatures=TeX,Scale=1}
\fi
% Use upquote if available, for straight quotes in verbatim environments
\IfFileExists{upquote.sty}{\usepackage{upquote}}{}
\IfFileExists{microtype.sty}{% use microtype if available
  \usepackage[]{microtype}
  \UseMicrotypeSet[protrusion]{basicmath} % disable protrusion for tt fonts
}{}
\makeatletter
\@ifundefined{KOMAClassName}{% if non-KOMA class
  \IfFileExists{parskip.sty}{%
    \usepackage{parskip}
  }{% else
    \setlength{\parindent}{0pt}
    \setlength{\parskip}{6pt plus 2pt minus 1pt}}
}{% if KOMA class
  \KOMAoptions{parskip=half}}
\makeatother
\usepackage{xcolor}
\usepackage[margin=1in]{geometry}
\usepackage{color}
\usepackage{fancyvrb}
\newcommand{\VerbBar}{|}
\newcommand{\VERB}{\Verb[commandchars=\\\{\}]}
\DefineVerbatimEnvironment{Highlighting}{Verbatim}{commandchars=\\\{\}}
% Add ',fontsize=\small' for more characters per line
\usepackage{framed}
\definecolor{shadecolor}{RGB}{248,248,248}
\newenvironment{Shaded}{\begin{snugshade}}{\end{snugshade}}
\newcommand{\AlertTok}[1]{\textcolor[rgb]{0.94,0.16,0.16}{#1}}
\newcommand{\AnnotationTok}[1]{\textcolor[rgb]{0.56,0.35,0.01}{\textbf{\textit{#1}}}}
\newcommand{\AttributeTok}[1]{\textcolor[rgb]{0.77,0.63,0.00}{#1}}
\newcommand{\BaseNTok}[1]{\textcolor[rgb]{0.00,0.00,0.81}{#1}}
\newcommand{\BuiltInTok}[1]{#1}
\newcommand{\CharTok}[1]{\textcolor[rgb]{0.31,0.60,0.02}{#1}}
\newcommand{\CommentTok}[1]{\textcolor[rgb]{0.56,0.35,0.01}{\textit{#1}}}
\newcommand{\CommentVarTok}[1]{\textcolor[rgb]{0.56,0.35,0.01}{\textbf{\textit{#1}}}}
\newcommand{\ConstantTok}[1]{\textcolor[rgb]{0.00,0.00,0.00}{#1}}
\newcommand{\ControlFlowTok}[1]{\textcolor[rgb]{0.13,0.29,0.53}{\textbf{#1}}}
\newcommand{\DataTypeTok}[1]{\textcolor[rgb]{0.13,0.29,0.53}{#1}}
\newcommand{\DecValTok}[1]{\textcolor[rgb]{0.00,0.00,0.81}{#1}}
\newcommand{\DocumentationTok}[1]{\textcolor[rgb]{0.56,0.35,0.01}{\textbf{\textit{#1}}}}
\newcommand{\ErrorTok}[1]{\textcolor[rgb]{0.64,0.00,0.00}{\textbf{#1}}}
\newcommand{\ExtensionTok}[1]{#1}
\newcommand{\FloatTok}[1]{\textcolor[rgb]{0.00,0.00,0.81}{#1}}
\newcommand{\FunctionTok}[1]{\textcolor[rgb]{0.00,0.00,0.00}{#1}}
\newcommand{\ImportTok}[1]{#1}
\newcommand{\InformationTok}[1]{\textcolor[rgb]{0.56,0.35,0.01}{\textbf{\textit{#1}}}}
\newcommand{\KeywordTok}[1]{\textcolor[rgb]{0.13,0.29,0.53}{\textbf{#1}}}
\newcommand{\NormalTok}[1]{#1}
\newcommand{\OperatorTok}[1]{\textcolor[rgb]{0.81,0.36,0.00}{\textbf{#1}}}
\newcommand{\OtherTok}[1]{\textcolor[rgb]{0.56,0.35,0.01}{#1}}
\newcommand{\PreprocessorTok}[1]{\textcolor[rgb]{0.56,0.35,0.01}{\textit{#1}}}
\newcommand{\RegionMarkerTok}[1]{#1}
\newcommand{\SpecialCharTok}[1]{\textcolor[rgb]{0.00,0.00,0.00}{#1}}
\newcommand{\SpecialStringTok}[1]{\textcolor[rgb]{0.31,0.60,0.02}{#1}}
\newcommand{\StringTok}[1]{\textcolor[rgb]{0.31,0.60,0.02}{#1}}
\newcommand{\VariableTok}[1]{\textcolor[rgb]{0.00,0.00,0.00}{#1}}
\newcommand{\VerbatimStringTok}[1]{\textcolor[rgb]{0.31,0.60,0.02}{#1}}
\newcommand{\WarningTok}[1]{\textcolor[rgb]{0.56,0.35,0.01}{\textbf{\textit{#1}}}}
\usepackage{longtable,booktabs,array}
\usepackage{calc} % for calculating minipage widths
% Correct order of tables after \paragraph or \subparagraph
\usepackage{etoolbox}
\makeatletter
\patchcmd\longtable{\par}{\if@noskipsec\mbox{}\fi\par}{}{}
\makeatother
% Allow footnotes in longtable head/foot
\IfFileExists{footnotehyper.sty}{\usepackage{footnotehyper}}{\usepackage{footnote}}
\makesavenoteenv{longtable}
\usepackage{graphicx}
\makeatletter
\def\maxwidth{\ifdim\Gin@nat@width>\linewidth\linewidth\else\Gin@nat@width\fi}
\def\maxheight{\ifdim\Gin@nat@height>\textheight\textheight\else\Gin@nat@height\fi}
\makeatother
% Scale images if necessary, so that they will not overflow the page
% margins by default, and it is still possible to overwrite the defaults
% using explicit options in \includegraphics[width, height, ...]{}
\setkeys{Gin}{width=\maxwidth,height=\maxheight,keepaspectratio}
% Set default figure placement to htbp
\makeatletter
\def\fps@figure{htbp}
\makeatother
\setlength{\emergencystretch}{3em} % prevent overfull lines
\providecommand{\tightlist}{%
  \setlength{\itemsep}{0pt}\setlength{\parskip}{0pt}}
\setcounter{secnumdepth}{-\maxdimen} % remove section numbering
\usepackage{hyperref}
\usepackage{array}
\usepackage{caption}
\usepackage{graphicx}
\usepackage{multirow}
\usepackage{hhline}
\usepackage{calc}
\usepackage{tabularx}
\usepackage[para,online,flushleft]{threeparttable}
\ifLuaTeX
  \usepackage{selnolig}  % disable illegal ligatures
\fi
\IfFileExists{bookmark.sty}{\usepackage{bookmark}}{\usepackage{hyperref}}
\IfFileExists{xurl.sty}{\usepackage{xurl}}{} % add URL line breaks if available
\urlstyle{same} % disable monospaced font for URLs
\hypersetup{
  pdftitle={Analysis of Health Survey for England (HSE) 2019},
  pdfauthor={Candidate Numbers Here},
  hidelinks,
  pdfcreator={LaTeX via pandoc}}

\title{Analysis of Health Survey for England (HSE) 2019}
\author{Candidate Numbers Here}
\date{March 02, 2024}

\begin{document}
\maketitle
\begin{abstract}
This report provides an analysis of data related to health, age,
socio-economic factors and lifestyle habits in adults (from the age of
16) from the population in England, derived from the Health Survey for
England 2019.
\end{abstract}

\newpage

\hypertarget{introduction}{%
\section{Introduction}\label{introduction}}

This is a body of text. \emph{This is an italic body of text.}
\href{https://google.com}{This is a clickable link!}.

\hypertarget{some-yaml-stuff}{%
\section{Some YAML Stuff}\label{some-yaml-stuff}}

The lion's share of a R Markdown document will be raw text, though the
front matter may be the most important part of the document. R Markdown
uses \href{http://www.yaml.org/}{YAML} for its metadata and the fields
differ from
\href{http://svmiller.com/blog/2015/02/moving-from-beamer-to-r-markdown/}{what
an author would use for a Beamer presentation}. I provide a sample YAML
metadata largely taken from this exact document and explain it below.

\begin{Shaded}
\begin{Highlighting}[]
\SpecialCharTok{{-}{-}{-}}
\NormalTok{output}\SpecialCharTok{:} 
\NormalTok{  pdf\_document}\SpecialCharTok{:}
\NormalTok{    keep\_tex}\SpecialCharTok{:}\NormalTok{ true}
\NormalTok{    fig\_caption}\SpecialCharTok{:}\NormalTok{ true}
\NormalTok{    latex\_engine}\SpecialCharTok{:}\NormalTok{ pdflatex}
\NormalTok{title}\SpecialCharTok{:} \StringTok{"A Pandoc Markdown Article Starter and Template"}
\NormalTok{abstract}\SpecialCharTok{:} \StringTok{"This document provides an introduction to R Markdown, argues for its..."}
\NormalTok{date}\SpecialCharTok{:} \StringTok{"\textasciigrave{}r format(Sys.time(), \textquotesingle{}\%B \%d, \%Y\textquotesingle{})\textasciigrave{}"}
\NormalTok{geometry}\SpecialCharTok{:}\NormalTok{ margin}\OtherTok{=}\NormalTok{1in}
\NormalTok{fontsize}\SpecialCharTok{:}\NormalTok{ 11pt}
\CommentTok{\# spacing: double}
\SpecialCharTok{{-}{-}{-}}
\end{Highlighting}
\end{Shaded}

\texttt{output:} will tell R Markdown we want a PDF document rendered
with LaTeX. Since we are adding a fair bit of custom options to this
call, we specify \texttt{pdf\_document:} on the next line (with,
importantly, a two-space indent). We specify additional output-level
options underneath it, each are indented with four spaces. The line
(\texttt{keep\_tex:\ true}) tells R Markdown to render a raw
\texttt{.tex} file along with the PDF document. This is useful for both
debugging and the publication stage. The next line
\texttt{fig\_caption:\ true} tells R Markdown to make sure that whatever
images are included in the document are treated as figures in which our
caption in brackets in a Markdown call is treated as the caption in the
figure. The next line (\texttt{latex\_engine:\ pdflatex}) tells R
Markdown to use pdflatex and not some other option like
\texttt{lualatex}. For this template, I'm pretty sure this is
mandatory.{[}\^{}pdflatex{]}

The next fields get to the heart of the document itself. \texttt{title:}
is, intuitively, the title of the manuscript. Do note that fields like
\texttt{title:} do not have to be in quotation marks, but must be in
quotation marks if the title of the document includes a colon. That
said, the only reason to use a colon in an article title is if it is
followed by a subtitle, hence the optional field (\texttt{subtitle:}).
Notice I ``comment out'' the subtitle in the above example with a pound
sign since this particular document does not have a subtitle.

\texttt{date} comes standard with R Markdown and you can use it to enter
the date of the most recent compile.

The next items are optional and cosmetic. \texttt{geometry:} is a
standard option in LaTeX. I set the margins at one inch, and you
probably should too. \texttt{fontsize:} sets, intuitively, the font
size. The default is 10-point, but I prefer 11-point. \texttt{spacing:}
is an optional field. If it is set as ``double'', the ensuing document
is double-spaced. ``single'' is the only other valid entry for this
field, though not including the entry in the YAML metadata amounts to
singlespacing the document by default. Notice I have this ``commented
out'' in the example code.

\hypertarget{getting-started-with-markdown-syntax}{%
\section{Getting Started with Markdown
Syntax}\label{getting-started-with-markdown-syntax}}

There are a lot of cheatsheets and reference guides for Markdown
(e.g.~\href{https://github.com/adam-p/markdown-here/wiki/Markdown-Cheatsheet}{Adam
Prichard},
\href{http://assemble.io/docs/Cheatsheet-Markdown.html}{Assemble},
\href{https://www.rstudio.com/wp-content/uploads/2015/02/rmarkdown-cheatsheet.pdf}{Rstudio},
\href{https://www.rstudio.com/wp-content/uploads/2015/03/rmarkdown-reference.pdf}{Rstudio
again},
\href{http://scottboms.com/downloads/documentation/markdown_cheatsheet.pdf}{Scott
Boms}, \href{https://daringfireball.net/projects/markdown/syntax}{Daring
Fireball}, among, I'm sure, several others).

\begin{Shaded}
\begin{Highlighting}[]

\FunctionTok{\# Introduction}

\NormalTok{**Lorem ipsum** dolor *sit amet*. }

\SpecialStringTok{{-} }\NormalTok{Single asterisks italicize text *like this*. }
\SpecialStringTok{{-} }\NormalTok{Double asterisks embolden text **like this**.}

\NormalTok{Start a new paragraph with a blank line separating paragraphs.}

\SpecialStringTok{{-} }\NormalTok{This will start an unordered list environment, and this will be the first item.}
\SpecialStringTok{{-} }\NormalTok{This will be a second item.}
\SpecialStringTok{{-} }\NormalTok{A third item.}
\SpecialStringTok{    {-} }\NormalTok{Four spaces and a dash create a sublist and this item in it.}
\SpecialStringTok{{-} }\NormalTok{The fourth item.}
    
\SpecialStringTok{1. }\NormalTok{This starts a numerical list.}
\SpecialStringTok{2. }\NormalTok{This is no. 2 in the numerical list.}
    
\FunctionTok{\# This Starts A New Section}
\FunctionTok{\#\# This is a Subsection}
\FunctionTok{\#\#\# This is a Subsubsection}
\FunctionTok{\#\#\#\# This starts a Paragraph Block.}

\AttributeTok{\textgreater{} This will create a block quote, if you want one.}

\NormalTok{Want a table? This will create one.}

\NormalTok{Table Header  | Second Header}
\NormalTok{{-}{-}{-}{-}{-}{-}{-}{-}{-}{-}{-}{-}{-} | {-}{-}{-}{-}{-}{-}{-}{-}{-}{-}{-}{-}{-}}
\NormalTok{Table Cell    | Cell 2}
\NormalTok{Cell 3        | Cell 4 }

\NormalTok{Note that the separators *do not* have to be aligned.}

\NormalTok{Want an image? This will do it.}

\AlertTok{![caption for my image](path/to/image.jpg)}

\InformationTok{\textasciigrave{}fig\_caption: yes\textasciigrave{}}\NormalTok{ will provide a caption. Put that in the YAML metadata.}

\NormalTok{Almost forgot about creating a footnote.}\OtherTok{[\^{}1]}\NormalTok{ This will do it again.}\OtherTok{[\^{}2]}

\OtherTok{[\^{}1]: }\NormalTok{The first footnote}
\OtherTok{[\^{}2]: }\NormalTok{The second footnote}

\NormalTok{Want to cite something? }

\SpecialStringTok{{-} }\NormalTok{Find your biblatexkey in your bib file.}
\SpecialStringTok{{-} }\NormalTok{Put an @ before it, like @smith1984, or whatever it is.}
\SpecialStringTok{{-} }\NormalTok{@smith1984 creates an in{-}text citation (e.g. Smith (1984) says...)}
\SpecialStringTok{{-} }\CommentTok{[}\OtherTok{@smith1984}\CommentTok{]}\NormalTok{ creates a parenthetical citation (Smith, 1984)}

\NormalTok{That\textquotesingle{}ll also automatically create a reference list at the end of the document.}

\CommentTok{[}\OtherTok{In{-}text link to Google}\CommentTok{](http://google.com)}\NormalTok{ as well.}
\end{Highlighting}
\end{Shaded}

\hypertarget{exploring-the-data}{%
\subsection{Exploring the Data}\label{exploring-the-data}}

\hypertarget{checking-for-messy-data}{%
\subsubsection{Checking for Messy Data}\label{checking-for-messy-data}}

\begin{Shaded}
\begin{Highlighting}[]
\FunctionTok{library}\NormalTok{(haven) }\CommentTok{\# Required to present the summary of labelled data.}
\FunctionTok{load}\NormalTok{(}\StringTok{"\textasciitilde{}/MA30091/Coursework/MA30091/Datasets/hsesub.Rdata"}\NormalTok{) }\CommentTok{\# The dset is called subdat}
\FunctionTok{summary}\NormalTok{(subdat)}
\end{Highlighting}
\end{Shaded}

\begin{verbatim}
##     SerialA             Sex           ag16g10          Age35g     
##  Min.   :2900001   Min.   :1.000   Min.   :1.000   Min.   : 1.00  
##  1st Qu.:2903094   1st Qu.:1.000   1st Qu.:3.000   1st Qu.: 8.00  
##  Median :2906238   Median :2.000   Median :4.000   Median :12.00  
##  Mean   :2906229   Mean   :1.539   Mean   :4.128   Mean   :11.71  
##  3rd Qu.:2909378   3rd Qu.:2.000   3rd Qu.:6.000   3rd Qu.:16.00  
##  Max.   :2912465   Max.   :2.000   Max.   :7.000   Max.   :22.00  
##                                    NA's   :2095                   
##      wt_int          topqual2        marstatD         qimd19     
##  Min.   :0.3155   Min.   :1.000   Min.   :1.000   Min.   :1.000  
##  1st Qu.:0.7941   1st Qu.:1.000   1st Qu.:2.000   1st Qu.:2.000  
##  Median :0.8989   Median :3.000   Median :2.000   Median :3.000  
##  Mean   :1.0000   Mean   :3.664   Mean   :2.658   Mean   :3.044  
##  3rd Qu.:1.0974   3rd Qu.:7.000   3rd Qu.:4.000   3rd Qu.:4.000  
##  Max.   :6.4927   Max.   :8.000   Max.   :6.000   Max.   :5.000  
##                   NA's   :2141    NA's   :2096                   
##     urban14b        origin2        cigsta3_19      cigdyal_19    
##  Min.   :1.000   Min.   :1.000   Min.   :1.000   Min.   : 0.000  
##  1st Qu.:1.000   1st Qu.:1.000   1st Qu.:2.000   1st Qu.: 0.000  
##  Median :1.000   Median :1.000   Median :3.000   Median : 0.000  
##  Mean   :1.181   Mean   :1.343   Mean   :2.437   Mean   : 1.692  
##  3rd Qu.:1.000   3rd Qu.:1.000   3rd Qu.:3.000   3rd Qu.: 0.000  
##  Max.   :2.000   Max.   :5.000   Max.   :3.000   Max.   :60.000  
##                  NA's   :33      NA's   :2151    NA's   :2152    
##      BMIVal         NDPNow_19        dnoft_19       drinkYN_19   
##  Min.   : 9.723   Min.   :1.000   Min.   :1.000   Min.   :1.000  
##  1st Qu.:21.915   1st Qu.:4.000   1st Qu.:3.000   1st Qu.:2.000  
##  Median :25.904   Median :4.000   Median :4.000   Median :2.000  
##  Mean   :26.223   Mean   :3.862   Mean   :4.281   Mean   :1.808  
##  3rd Qu.:29.953   3rd Qu.:4.000   3rd Qu.:5.000   3rd Qu.:2.000  
##  Max.   :73.494   Max.   :4.000   Max.   :8.000   Max.   :2.000  
##  NA's   :2224     NA's   :2148    NA's   :3594    NA's   :2146   
##    d7many3_19       omsysval          GOR1      
##  Min.   :0.000   Min.   : 75.0   Min.   :1.000  
##  1st Qu.:0.000   1st Qu.:110.5   1st Qu.:3.000  
##  Median :1.000   Median :121.0   Median :5.000  
##  Mean   :1.595   Mean   :122.9   Mean   :5.163  
##  3rd Qu.:3.000   3rd Qu.:133.5   3rd Qu.:8.000  
##  Max.   :7.000   Max.   :209.5   Max.   :9.000  
##  NA's   :2147    NA's   :5593
\end{verbatim}

This tells us that all of our variables are coded as numeric. However,
we may want to code some as factor variables instead based on the
variable descriptions.

\begin{itemize}
\tightlist
\item
  Sex: Should be coded as
\end{itemize}

\begin{longtable}[]{@{}lll@{}}
\toprule()
Code & Decode & Count \\
\midrule()
\endhead
1 & Male & \\
2 & Female & \\
-1 & Not Applicable & \\
-8 & Don't Know & \\
-9 & Refused & \\
\bottomrule()
\end{longtable}

\begin{itemize}
\tightlist
\item
  Age35g: Should be coded as
\end{itemize}

\begin{longtable}[]{@{}lll@{}}
\toprule()
Code & Decode & Count \\
\midrule()
\endhead
1 & 0-1yrs & \\
2 & 2-4yrs & \\
3 & 5-7yrs & \\
4 & 8-10yrs & \\
5 & 11-12yrs & \\
6 & 13-15yrs & \\
7 & 16-19yrs & \\
8 & 20-24yrs & \\
9 & 25-29yrs & \\
10 & 30-34yrs & \\
11 & 35-39yrs & \\
12 & 40-44yrs & \\
13 & 45-49yrs & \\
14 & 50-54yrs & \\
15 & 55-59yrs & \\
16 & 60-64yrs & \\
17 & 65-69yrs & \\
18 & 70-74yrs & \\
19 & 75-79yrs & \\
20 & 80-84yrs & \\
21 & 85-59yrs & \\
22 & 90+yrs & \\
-1 & Not Applicable & \\
-8 & Don't Know & \\
-9 & Refused & \\
\bottomrule()
\end{longtable}

\begin{itemize}
\tightlist
\item
  ag16g10: Should be coded as
\end{itemize}

\begin{longtable}[]{@{}lll@{}}
\toprule()
Code & Decode & Count \\
\midrule()
\endhead
1 & 16-24yrs & \\
2 & 25-34yrs & \\
3 & 35-44yrs & \\
4 & 45-54yrs & \\
5 & 55-64yrs & \\
6 & 65-74yrs & \\
7 & 75+yrs & \\
-1 & Not Applicable & \\
-8 & Don't Know & \\
-9 & Refused & \\
\bottomrule()
\end{longtable}

\begin{itemize}
\tightlist
\item
  topqual2: Should be coded as
\end{itemize}

\begin{longtable}[]{@{}lll@{}}
\toprule()
Code & Decode & Count \\
\midrule()
\endhead
1 & NVQ4/NVQ5/Degree or equiv & \\
2 & Higher ed below degree & \\
3 & NVQ3/GCE A Level equiv & \\
4 & NVQ2/GCE O Level equiv & \\
5 & NVQ1/CSE other grade equiv & \\
6 & Foreign/other & \\
7 & No qualification & \\
8 & FT Student & \\
-1 & Not Applicable & \\
-8 & Don't Know & \\
-9 & Refused & \\
\bottomrule()
\end{longtable}

\begin{itemize}
\tightlist
\item
  qimd19: Should be coded as
\end{itemize}

\begin{longtable}[]{@{}lll@{}}
\toprule()
Code & Decode & Count \\
\midrule()
\endhead
1 & Most deprived & \\
5 & Least deprived & \\
-1 & Not Applicable & \\
-8 & Don't Know & \\
-9 & Refused & \\
\bottomrule()
\end{longtable}

Note: IMD2,IMD3 and IMD4 had no observations.

\begin{itemize}
\tightlist
\item
  urban14b: Should be coded as
\end{itemize}

\begin{longtable}[]{@{}lll@{}}
\toprule()
Code & Decode & Count \\
\midrule()
\endhead
1 & Urban & \\
2 & Town/ Fringe/ Village, hamlet and isolated dwellings & \\
-1 & Not Applicable & \\
-8 & Don't Know & \\
-9 & Refused & \\
\bottomrule()
\end{longtable}

\begin{itemize}
\tightlist
\item
  origin2: Should be coded as
\end{itemize}

\begin{longtable}[]{@{}lll@{}}
\toprule()
Code & Decode & Count \\
\midrule()
\endhead
1 & White & \\
2 & Black & \\
3 & Asian & \\
4 & Mixed/multiple ethnic background & \\
5 & Any other ethnic group & \\
-1 & Not Applicable & \\
-8 & Don't Know & \\
-9 & Refused & \\
\bottomrule()
\end{longtable}

\begin{itemize}
\tightlist
\item
  cigsta3\_19: Should be coded as
\end{itemize}

\begin{longtable}[]{@{}lll@{}}
\toprule()
Code & Decode & Count \\
\midrule()
\endhead
1 & Current cigarette smoker & \\
2 & Ex-regular cigarette smoker & \\
3 & Never regular cigarette smoker & \\
-1 & Not Applicable & \\
-8 & Don't Know & \\
-9 & Refused & \\
\bottomrule()
\end{longtable}

\begin{itemize}
\tightlist
\item
  NDPNow\_19: Should be coded as
\end{itemize}

\begin{longtable}[]{@{}lll@{}}
\toprule()
Code & Decode & Count \\
\midrule()
\endhead
1 & E-cigarettes or vaping devices only & \\
2 & Other nicotine delivery products only & \\
3 & Both & \\
4 & None & \\
-1 & Not Applicable & \\
-8 & Don't Know & \\
-9 & Refused & \\
\bottomrule()
\end{longtable}

\begin{itemize}
\tightlist
\item
  drinkYN\_19: Should be coded as
\end{itemize}

\begin{longtable}[]{@{}lll@{}}
\toprule()
Code & Decode & Count \\
\midrule()
\endhead
1 & No & \\
2 & Yes & \\
-1 & Not Applicable & \\
-8 & Don't Know & \\
-9 & Refused & \\
\bottomrule()
\end{longtable}

\begin{itemize}
\tightlist
\item
  dnoft\_19: Should be coded as
\end{itemize}

\begin{longtable}[]{@{}lll@{}}
\toprule()
Code & Decode & Count \\
\midrule()
\endhead
1 & Almost every day & \\
2 & Five or six days a week & \\
3 & Three or four days a week & \\
4 & Once or twice a week & \\
5 & Once or twice a month & \\
6 & Once every couple of months & \\
7 & Once or twice a year & \\
8 & Not at all in the last 12 months & \\
-1 & Not Applicable & \\
-8 & Don't Know & \\
-9 & Refused & \\
\bottomrule()
\end{longtable}

\begin{itemize}
\tightlist
\item
  GOR1: Should be coded as
\end{itemize}

\begin{longtable}[]{@{}lll@{}}
\toprule()
Code & Decode & Count \\
\midrule()
\endhead
1 & North East & \\
2 & North West & \\
3 & Yorkshire and the Humber & \\
4 & East Midlands & \\
5 & West Midlands & \\
6 & East of England & \\
7 & London & \\
8 & South East & \\
9 & South West & \\
-1 & Not Applicable & \\
-8 & Don't Know & \\
-9 & Refused & \\
\bottomrule()
\end{longtable}

\begin{Shaded}
\begin{Highlighting}[]
\NormalTok{subdat}\SpecialCharTok{$}\NormalTok{Sex }\OtherTok{=} \FunctionTok{factor}\NormalTok{(subdat}\SpecialCharTok{$}\NormalTok{Sex)}
\NormalTok{subdat}\SpecialCharTok{$}\NormalTok{Age35g }\OtherTok{=} \FunctionTok{factor}\NormalTok{(subdat}\SpecialCharTok{$}\NormalTok{Age35g)}
\NormalTok{subdat}\SpecialCharTok{$}\NormalTok{ag16g10 }\OtherTok{=} \FunctionTok{factor}\NormalTok{(subdat}\SpecialCharTok{$}\NormalTok{ag16g10)}
\NormalTok{subdat}\SpecialCharTok{$}\NormalTok{topqual2 }\OtherTok{=} \FunctionTok{factor}\NormalTok{(subdat}\SpecialCharTok{$}\NormalTok{topqual2)}
\NormalTok{subdat}\SpecialCharTok{$}\NormalTok{qimd19 }\OtherTok{=} \FunctionTok{factor}\NormalTok{(subdat}\SpecialCharTok{$}\NormalTok{qimd19)}
\NormalTok{subdat}\SpecialCharTok{$}\NormalTok{urban14b }\OtherTok{=} \FunctionTok{factor}\NormalTok{(subdat}\SpecialCharTok{$}\NormalTok{urban14b)}
\NormalTok{subdat}\SpecialCharTok{$}\NormalTok{origin2 }\OtherTok{=} \FunctionTok{factor}\NormalTok{(subdat}\SpecialCharTok{$}\NormalTok{origin2)}
\NormalTok{subdat}\SpecialCharTok{$}\NormalTok{cigsta3\_19 }\OtherTok{=} \FunctionTok{factor}\NormalTok{(subdat}\SpecialCharTok{$}\NormalTok{cigsta3\_19)}
\NormalTok{subdat}\SpecialCharTok{$}\NormalTok{NDPNow\_19 }\OtherTok{=} \FunctionTok{factor}\NormalTok{(subdat}\SpecialCharTok{$}\NormalTok{NDPNow\_19)}
\NormalTok{subdat}\SpecialCharTok{$}\NormalTok{drinkYN\_19 }\OtherTok{=} \FunctionTok{factor}\NormalTok{(subdat}\SpecialCharTok{$}\NormalTok{drinkYN\_19)}
\NormalTok{subdat}\SpecialCharTok{$}\NormalTok{dnoft\_19 }\OtherTok{=} \FunctionTok{factor}\NormalTok{(subdat}\SpecialCharTok{$}\NormalTok{dnoft\_19)}
\NormalTok{subdat}\SpecialCharTok{$}\NormalTok{GOR1 }\OtherTok{=} \FunctionTok{factor}\NormalTok{(subdat}\SpecialCharTok{$}\NormalTok{GOR1)}
\FunctionTok{summary}\NormalTok{(subdat)}
\end{Highlighting}
\end{Shaded}

\begin{verbatim}
##     SerialA        Sex         ag16g10         Age35g         wt_int      
##  Min.   :2900001   1:4745   4      :1416   14     : 735   Min.   :0.3155  
##  1st Qu.:2903094   2:5554   3      :1397   11     : 725   1st Qu.:0.7941  
##  Median :2906238            5      :1349   15     : 693   Median :0.8989  
##  Mean   :2906229            6      :1242   13     : 681   Mean   :1.0000  
##  3rd Qu.:2909378            2      :1083   12     : 672   3rd Qu.:1.0974  
##  Max.   :2912465            (Other):1717   16     : 656   Max.   :6.4927  
##                             NA's   :2095   (Other):6137                   
##     topqual2       marstatD     qimd19   urban14b origin2     cigsta3_19 
##  1      :2320   Min.   :1.000   1:2074   1:8433   1   :8561   1   :1254  
##  7      :1616   1st Qu.:2.000   2:1942   2:1866   2   : 345   2   :2076  
##  4      :1432   Median :2.000   3:1965            3   :1007   3   :4818  
##  3      :1106   Mean   :2.658   4:2091            4   : 250   NA's:2151  
##  2      : 873   3rd Qu.:4.000   5:2227            5   : 103              
##  (Other): 811   Max.   :6.000                     NA's:  33              
##  NA's   :2141   NA's   :2096                                             
##    cigdyal_19         BMIVal       NDPNow_19      dnoft_19    drinkYN_19 
##  Min.   : 0.000   Min.   : 9.723   1   : 317   4      :1978   1   :1567  
##  1st Qu.: 0.000   1st Qu.:21.915   2   :  78   5      :1191   2   :6586  
##  Median : 0.000   Median :25.904   3   :  17   3      :1106   NA's:2146  
##  Mean   : 1.692   Mean   :26.223   4   :7739   6      : 748              
##  3rd Qu.: 0.000   3rd Qu.:29.953   NA's:2148   7      : 705              
##  Max.   :60.000   Max.   :73.494               (Other): 977              
##  NA's   :2152     NA's   :2224                 NA's   :3594              
##    d7many3_19       omsysval          GOR1     
##  Min.   :0.000   Min.   : 75.0   8      :1620  
##  1st Qu.:0.000   1st Qu.:110.5   2      :1379  
##  Median :1.000   Median :121.0   7      :1284  
##  Mean   :1.595   Mean   :122.9   6      :1179  
##  3rd Qu.:3.000   3rd Qu.:133.5   3      :1138  
##  Max.   :7.000   Max.   :209.5   5      : 972  
##  NA's   :2147    NA's   :5593    (Other):2727
\end{verbatim}

Note that the null flavors may not be used for modeling (and can just be
treated as generic missing values), but they will bve useful for
evaluating the study design. For example, lots of \textbf{Refused} for a
variable could mean there is a bias in porivacy or that the question is
too sensitive. Lots of \textbf{Don't know} for a variable could indicate
some recall bias and that the question is poorly designed, whereas lots
of \textbf{Not applicable} either comes from reduced generalisability
(e.g.~``Is patient currently pregnant?) or poorly measured variables
(Like valid BMI results being sparse due to bad measurements or missing
heights/weights).

The variable d7many3\_19 has nothing but missing values, so this
variable can be dropped from analysis.

\end{document}
