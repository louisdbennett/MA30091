% Options for packages loaded elsewhere
\PassOptionsToPackage{unicode}{hyperref}
\PassOptionsToPackage{hyphens}{url}
%
\documentclass[
  11pt,
]{article}
\usepackage{amsmath,amssymb}
\usepackage{lmodern}
\usepackage{iftex}
\ifPDFTeX
  \usepackage[T1]{fontenc}
  \usepackage[utf8]{inputenc}
  \usepackage{textcomp} % provide euro and other symbols
\else % if luatex or xetex
  \usepackage{unicode-math}
  \defaultfontfeatures{Scale=MatchLowercase}
  \defaultfontfeatures[\rmfamily]{Ligatures=TeX,Scale=1}
\fi
% Use upquote if available, for straight quotes in verbatim environments
\IfFileExists{upquote.sty}{\usepackage{upquote}}{}
\IfFileExists{microtype.sty}{% use microtype if available
  \usepackage[]{microtype}
  \UseMicrotypeSet[protrusion]{basicmath} % disable protrusion for tt fonts
}{}
\makeatletter
\@ifundefined{KOMAClassName}{% if non-KOMA class
  \IfFileExists{parskip.sty}{%
    \usepackage{parskip}
  }{% else
    \setlength{\parindent}{0pt}
    \setlength{\parskip}{6pt plus 2pt minus 1pt}}
}{% if KOMA class
  \KOMAoptions{parskip=half}}
\makeatother
\usepackage{xcolor}
\usepackage[margin=1in]{geometry}
\usepackage{color}
\usepackage{fancyvrb}
\newcommand{\VerbBar}{|}
\newcommand{\VERB}{\Verb[commandchars=\\\{\}]}
\DefineVerbatimEnvironment{Highlighting}{Verbatim}{commandchars=\\\{\}}
% Add ',fontsize=\small' for more characters per line
\usepackage{framed}
\definecolor{shadecolor}{RGB}{248,248,248}
\newenvironment{Shaded}{\begin{snugshade}}{\end{snugshade}}
\newcommand{\AlertTok}[1]{\textcolor[rgb]{0.94,0.16,0.16}{#1}}
\newcommand{\AnnotationTok}[1]{\textcolor[rgb]{0.56,0.35,0.01}{\textbf{\textit{#1}}}}
\newcommand{\AttributeTok}[1]{\textcolor[rgb]{0.77,0.63,0.00}{#1}}
\newcommand{\BaseNTok}[1]{\textcolor[rgb]{0.00,0.00,0.81}{#1}}
\newcommand{\BuiltInTok}[1]{#1}
\newcommand{\CharTok}[1]{\textcolor[rgb]{0.31,0.60,0.02}{#1}}
\newcommand{\CommentTok}[1]{\textcolor[rgb]{0.56,0.35,0.01}{\textit{#1}}}
\newcommand{\CommentVarTok}[1]{\textcolor[rgb]{0.56,0.35,0.01}{\textbf{\textit{#1}}}}
\newcommand{\ConstantTok}[1]{\textcolor[rgb]{0.00,0.00,0.00}{#1}}
\newcommand{\ControlFlowTok}[1]{\textcolor[rgb]{0.13,0.29,0.53}{\textbf{#1}}}
\newcommand{\DataTypeTok}[1]{\textcolor[rgb]{0.13,0.29,0.53}{#1}}
\newcommand{\DecValTok}[1]{\textcolor[rgb]{0.00,0.00,0.81}{#1}}
\newcommand{\DocumentationTok}[1]{\textcolor[rgb]{0.56,0.35,0.01}{\textbf{\textit{#1}}}}
\newcommand{\ErrorTok}[1]{\textcolor[rgb]{0.64,0.00,0.00}{\textbf{#1}}}
\newcommand{\ExtensionTok}[1]{#1}
\newcommand{\FloatTok}[1]{\textcolor[rgb]{0.00,0.00,0.81}{#1}}
\newcommand{\FunctionTok}[1]{\textcolor[rgb]{0.00,0.00,0.00}{#1}}
\newcommand{\ImportTok}[1]{#1}
\newcommand{\InformationTok}[1]{\textcolor[rgb]{0.56,0.35,0.01}{\textbf{\textit{#1}}}}
\newcommand{\KeywordTok}[1]{\textcolor[rgb]{0.13,0.29,0.53}{\textbf{#1}}}
\newcommand{\NormalTok}[1]{#1}
\newcommand{\OperatorTok}[1]{\textcolor[rgb]{0.81,0.36,0.00}{\textbf{#1}}}
\newcommand{\OtherTok}[1]{\textcolor[rgb]{0.56,0.35,0.01}{#1}}
\newcommand{\PreprocessorTok}[1]{\textcolor[rgb]{0.56,0.35,0.01}{\textit{#1}}}
\newcommand{\RegionMarkerTok}[1]{#1}
\newcommand{\SpecialCharTok}[1]{\textcolor[rgb]{0.00,0.00,0.00}{#1}}
\newcommand{\SpecialStringTok}[1]{\textcolor[rgb]{0.31,0.60,0.02}{#1}}
\newcommand{\StringTok}[1]{\textcolor[rgb]{0.31,0.60,0.02}{#1}}
\newcommand{\VariableTok}[1]{\textcolor[rgb]{0.00,0.00,0.00}{#1}}
\newcommand{\VerbatimStringTok}[1]{\textcolor[rgb]{0.31,0.60,0.02}{#1}}
\newcommand{\WarningTok}[1]{\textcolor[rgb]{0.56,0.35,0.01}{\textbf{\textit{#1}}}}
\usepackage{graphicx}
\makeatletter
\def\maxwidth{\ifdim\Gin@nat@width>\linewidth\linewidth\else\Gin@nat@width\fi}
\def\maxheight{\ifdim\Gin@nat@height>\textheight\textheight\else\Gin@nat@height\fi}
\makeatother
% Scale images if necessary, so that they will not overflow the page
% margins by default, and it is still possible to overwrite the defaults
% using explicit options in \includegraphics[width, height, ...]{}
\setkeys{Gin}{width=\maxwidth,height=\maxheight,keepaspectratio}
% Set default figure placement to htbp
\makeatletter
\def\fps@figure{htbp}
\makeatother
\setlength{\emergencystretch}{3em} % prevent overfull lines
\providecommand{\tightlist}{%
  \setlength{\itemsep}{0pt}\setlength{\parskip}{0pt}}
\setcounter{secnumdepth}{-\maxdimen} % remove section numbering
\usepackage{hyperref}
\usepackage{array}
\usepackage{caption}
\usepackage{graphicx}
\usepackage{multirow}
\usepackage{hhline}
\usepackage{calc}
\usepackage{tabularx}
\usepackage[para,online,flushleft]{threeparttable}
\ifLuaTeX
  \usepackage{selnolig}  % disable illegal ligatures
\fi
\IfFileExists{bookmark.sty}{\usepackage{bookmark}}{\usepackage{hyperref}}
\IfFileExists{xurl.sty}{\usepackage{xurl}}{} % add URL line breaks if available
\urlstyle{same} % disable monospaced font for URLs
\hypersetup{
  pdftitle={Analysis of Health Survey for England (HSE) 2019},
  pdfauthor={Candidate Numbers Here},
  hidelinks,
  pdfcreator={LaTeX via pandoc}}

\title{Analysis of Health Survey for England (HSE) 2019}
\author{Candidate Numbers Here}
\date{March 01, 2024}

\begin{document}
\maketitle
\begin{abstract}
This report provides an analysis of data related to health, age,
socio-economic factors and lifestyle habits in adults (from the age of
16) from the population in England, derived from the Health Survey for
England 2019.
\end{abstract}

\newpage

\hypertarget{introduction}{%
\section{Introduction}\label{introduction}}

This is a body of text. \emph{This is an italic body of text.}
\href{https://google.com}{This is a clickable link!}.

\hypertarget{some-yaml-stuff}{%
\section{Some YAML Stuff}\label{some-yaml-stuff}}

The lion's share of a R Markdown document will be raw text, though the
front matter may be the most important part of the document. R Markdown
uses \href{http://www.yaml.org/}{YAML} for its metadata and the fields
differ from
\href{http://svmiller.com/blog/2015/02/moving-from-beamer-to-r-markdown/}{what
an author would use for a Beamer presentation}. I provide a sample YAML
metadata largely taken from this exact document and explain it below.

\begin{Shaded}
\begin{Highlighting}[]
\SpecialCharTok{{-}{-}{-}}
\NormalTok{output}\SpecialCharTok{:} 
\NormalTok{  pdf\_document}\SpecialCharTok{:}
\NormalTok{    keep\_tex}\SpecialCharTok{:}\NormalTok{ true}
\NormalTok{    fig\_caption}\SpecialCharTok{:}\NormalTok{ true}
\NormalTok{    latex\_engine}\SpecialCharTok{:}\NormalTok{ pdflatex}
\NormalTok{title}\SpecialCharTok{:} \StringTok{"A Pandoc Markdown Article Starter and Template"}
\NormalTok{abstract}\SpecialCharTok{:} \StringTok{"This document provides an introduction to R Markdown, argues for its..."}
\NormalTok{date}\SpecialCharTok{:} \StringTok{"\textasciigrave{}r format(Sys.time(), \textquotesingle{}\%B \%d, \%Y\textquotesingle{})\textasciigrave{}"}
\NormalTok{geometry}\SpecialCharTok{:}\NormalTok{ margin}\OtherTok{=}\NormalTok{1in}
\NormalTok{fontsize}\SpecialCharTok{:}\NormalTok{ 11pt}
\CommentTok{\# spacing: double}
\SpecialCharTok{{-}{-}{-}}
\end{Highlighting}
\end{Shaded}

\texttt{output:} will tell R Markdown we want a PDF document rendered
with LaTeX. Since we are adding a fair bit of custom options to this
call, we specify \texttt{pdf\_document:} on the next line (with,
importantly, a two-space indent). We specify additional output-level
options underneath it, each are indented with four spaces. The line
(\texttt{keep\_tex:\ true}) tells R Markdown to render a raw
\texttt{.tex} file along with the PDF document. This is useful for both
debugging and the publication stage. The next line
\texttt{fig\_caption:\ true} tells R Markdown to make sure that whatever
images are included in the document are treated as figures in which our
caption in brackets in a Markdown call is treated as the caption in the
figure. The next line (\texttt{latex\_engine:\ pdflatex}) tells R
Markdown to use pdflatex and not some other option like
\texttt{lualatex}. For this template, I'm pretty sure this is
mandatory.{[}\^{}pdflatex{]}

The next fields get to the heart of the document itself. \texttt{title:}
is, intuitively, the title of the manuscript. Do note that fields like
\texttt{title:} do not have to be in quotation marks, but must be in
quotation marks if the title of the document includes a colon. That
said, the only reason to use a colon in an article title is if it is
followed by a subtitle, hence the optional field (\texttt{subtitle:}).
Notice I ``comment out'' the subtitle in the above example with a pound
sign since this particular document does not have a subtitle.

\texttt{date} comes standard with R Markdown and you can use it to enter
the date of the most recent compile.

The next items are optional and cosmetic. \texttt{geometry:} is a
standard option in LaTeX. I set the margins at one inch, and you
probably should too. \texttt{fontsize:} sets, intuitively, the font
size. The default is 10-point, but I prefer 11-point. \texttt{spacing:}
is an optional field. If it is set as ``double'', the ensuing document
is double-spaced. ``single'' is the only other valid entry for this
field, though not including the entry in the YAML metadata amounts to
singlespacing the document by default. Notice I have this ``commented
out'' in the example code.

\hypertarget{getting-started-with-markdown-syntax}{%
\section{Getting Started with Markdown
Syntax}\label{getting-started-with-markdown-syntax}}

There are a lot of cheatsheets and reference guides for Markdown
(e.g.~\href{https://github.com/adam-p/markdown-here/wiki/Markdown-Cheatsheet}{Adam
Prichard},
\href{http://assemble.io/docs/Cheatsheet-Markdown.html}{Assemble},
\href{https://www.rstudio.com/wp-content/uploads/2015/02/rmarkdown-cheatsheet.pdf}{Rstudio},
\href{https://www.rstudio.com/wp-content/uploads/2015/03/rmarkdown-reference.pdf}{Rstudio
again},
\href{http://scottboms.com/downloads/documentation/markdown_cheatsheet.pdf}{Scott
Boms}, \href{https://daringfireball.net/projects/markdown/syntax}{Daring
Fireball}, among, I'm sure, several others).

\begin{Shaded}
\begin{Highlighting}[]

\FunctionTok{\# Introduction}

\NormalTok{**Lorem ipsum** dolor *sit amet*. }

\SpecialStringTok{{-} }\NormalTok{Single asterisks italicize text *like this*. }
\SpecialStringTok{{-} }\NormalTok{Double asterisks embolden text **like this**.}

\NormalTok{Start a new paragraph with a blank line separating paragraphs.}

\SpecialStringTok{{-} }\NormalTok{This will start an unordered list environment, and this will be the first item.}
\SpecialStringTok{{-} }\NormalTok{This will be a second item.}
\SpecialStringTok{{-} }\NormalTok{A third item.}
\SpecialStringTok{    {-} }\NormalTok{Four spaces and a dash create a sublist and this item in it.}
\SpecialStringTok{{-} }\NormalTok{The fourth item.}
    
\SpecialStringTok{1. }\NormalTok{This starts a numerical list.}
\SpecialStringTok{2. }\NormalTok{This is no. 2 in the numerical list.}
    
\FunctionTok{\# This Starts A New Section}
\FunctionTok{\#\# This is a Subsection}
\FunctionTok{\#\#\# This is a Subsubsection}
\FunctionTok{\#\#\#\# This starts a Paragraph Block.}

\AttributeTok{\textgreater{} This will create a block quote, if you want one.}

\NormalTok{Want a table? This will create one.}

\NormalTok{Table Header  | Second Header}
\NormalTok{{-}{-}{-}{-}{-}{-}{-}{-}{-}{-}{-}{-}{-} | {-}{-}{-}{-}{-}{-}{-}{-}{-}{-}{-}{-}{-}}
\NormalTok{Table Cell    | Cell 2}
\NormalTok{Cell 3        | Cell 4 }

\NormalTok{Note that the separators *do not* have to be aligned.}

\NormalTok{Want an image? This will do it.}

\AlertTok{![caption for my image](path/to/image.jpg)}

\InformationTok{\textasciigrave{}fig\_caption: yes\textasciigrave{}}\NormalTok{ will provide a caption. Put that in the YAML metadata.}

\NormalTok{Almost forgot about creating a footnote.}\OtherTok{[\^{}1]}\NormalTok{ This will do it again.}\OtherTok{[\^{}2]}

\OtherTok{[\^{}1]: }\NormalTok{The first footnote}
\OtherTok{[\^{}2]: }\NormalTok{The second footnote}

\NormalTok{Want to cite something? }

\SpecialStringTok{{-} }\NormalTok{Find your biblatexkey in your bib file.}
\SpecialStringTok{{-} }\NormalTok{Put an @ before it, like @smith1984, or whatever it is.}
\SpecialStringTok{{-} }\NormalTok{@smith1984 creates an in{-}text citation (e.g. Smith (1984) says...)}
\SpecialStringTok{{-} }\CommentTok{[}\OtherTok{@smith1984}\CommentTok{]}\NormalTok{ creates a parenthetical citation (Smith, 1984)}

\NormalTok{That\textquotesingle{}ll also automatically create a reference list at the end of the document.}

\CommentTok{[}\OtherTok{In{-}text link to Google}\CommentTok{](http://google.com)}\NormalTok{ as well.}
\end{Highlighting}
\end{Shaded}


\end{document}
